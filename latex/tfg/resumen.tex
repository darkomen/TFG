
\thispagestyle{empty}
\begin{center}
    \Large
    \vspace{0.9cm}
    \textbf{Resumen}
\end{center}

El filamento usado en las impresoras 3D es creado mediante el proceso de fabricación por extrusión. Este proceso es utilizado habitualmente cuando se fabrican elementos con un perfil constante. A pesar de que la técnica está asentada en la industria, la extrusión de filamento es novedoso, ya que se deben de tener en cuenta muchos factores que a la hora de fabricar otros perfiles no eran importantes.\\

Estos factores afectan directamente a la calidad del hilo pudiendo ocasionar serios daños en las impresoras 3D si la calidad es baja. Hasta ahora, en las líneas de extrusión, los parámetros de fabricación son cualitativos, por ello el operador tan solo visualiza los parámetros para que en el instante de fabricación no haya ningún problema. Al fabricar un filamento que va a ser re-fundido es muy importante que todos los datos de produccion como por ejemplo temperaturas y diámetros del filamento, queden registrados, puesto que si no cumplen unos requisitos de calidad puede ocasionar daños al usuar y de esta forma podriamos analizar las posibles causas.\\

Este proyecto presenta un sistema de adquisición de datos con la posibilidad de acceder a la información de los parámetros de producción y almacenarla en un medio físico. De esta manera es posible registrar los datos con los que se ha generado el filamento y averiguar cuales son las causas de los fallos. Estos datos son tratados para generar informes detallados de las producciones, y tener un completo historial de la fabricación del filamento, así como un control de calidad del filamento producido. Con los datos adquiridos por el sistema hemos podido comprobar el funcionamiento de la extrusora, y en caso de detectar un error, hemos sido capaces de encontrar los motivos ayudándonos del mismo. También hemos sido capaces de realizar un estudio de la influencia de un regulador experto en el diámetro del filamento demostrando que los datos obtenidos, con nuestro sistema de adquisición de datos, nos han sido útiles para poder obtener un modelo parcial de nuestro sistema.\\

La totalidad del proyecto ha sido realizado en la empresa BQ. Se han abierto distintas líneas de trabajo, en las que se incluye un estudio detallado de como con el paso del tiempo se puede llegar a degradar el PLA y afectar a su correcto funcionamiento. La información descrita en esta memoria es el primer paso para crear un sistema de gestión de calidad de filamento que desde BQ usarán en un futuro próximo. \\

%\newpage{}
%\begin{center}
%    \Large
%    \vspace{0.9cm}
%    \textbf{Abstract}
%\end{center}
