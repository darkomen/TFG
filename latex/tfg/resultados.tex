\chapter{Resultados}
\label{cap:resultados}

Como se ha mostrado hasta ahora, se tiene un sistema capaz de registrar la información de una serie de sensores y crear un fichero con esa información. Cuando se incia el planteamiento de este proyecto, el paso final del mismo es implementar el sistema en una linea de extrusión industrial sin embargo, por problemas burocráticos, a la hora de implementar el funcionamiento del mismo, no se tiene acceso a ninguna. \\

Para poder demostrar que el sistema es útil, se decide hacer una maqueta de una extrusora para poder comprobar cómo con el sistema diseñado podemos realizar un estudio avanzado de los problemas que tenemos a la hora de fabricar filamento.\\


\section{Filastruder-sensor diámetro-bobinadora}
\label{sec:FSB}

Como primera aproximación e intentando asemejar el esquema de producción que tienen en la fábrica de Huesca se va a seguir el esquema mostrado en la figura \ref{fig:esquemap_FSB}:

\begin{figure}[H]
    \centering
    \includegraphics[width=0.6\textwidth]{images/producciones/Diagram1.png}
    \caption[Esquema de producción Filastruder-sensor de diámetro-bobinadora.]{Esquema de producción Filastruder-sensor de diámetro-bobinador.}
    \label{fig:esquemap_FSB}
\end{figure}

La boquilla de la filastruder es de 3mm, de esta manera se tiene margen suficiente para regular el filamento aplicando una fuerza de tracción según sale de la extrusora y se va enfriando. A la hora de trabajar con una técnica de conformado como es esta, es muy importante que la velocidad de salida sea lo más constante posible y también hay que tener en cuenta que se producen cambios en el material tanto de tamaño como de forma según sale de la extrusora. Tres de estos cambios que afectan al material son: \cite{tecno_polimeros}

\begin{itemize}
	\item {\textbf{Tensionado}: Habitualmente, el material es recogido por sistemas de almacenamiento consistentes en rodillos que aplican una tensión. Esto hace que el tamaño del material varié en función de esa tensión. Nos ayudaremos de esta característica para intentar regular el diámetro final del filamento. El tamaño del diámetro tendrá una relación inversamente proporcional a la velocidad.}
	\item {\textbf{Relajación}: Dentro de la extrusora, el material soprta esfuerzos normales y  al salir por la boqquilla se relaja. Esta relajación será mayor, en función del gradiente de temperatura que haya entre la boquila y el ambiente.
		\begin{figure}[H]
	        \centering
	        \includegraphics[width=0.2\textwidth]{images/producciones/relajacion.png}
	        \caption[Cambio de tamaño debido a la relajación del material.]{Cambio de tamaño debido a la relajación del material. Fuente \cite{tecno_polimeros}}
	        \label{fig:prod_relajacion}
		\end{figure}
	}
	\item {\textbf{Enfriamiento}: A medida que el material se va enfriando, se va generando una contracción en el perfil del mismo. Está contraccion depende de la velocidad de enfriamiento del material. Por ello, no habrá la misma contracción en una zona gruesa donde haya más material, que en una fina.
		\begin{figure}[H]
	        \centering
	        \includegraphics[width=0.2\textwidth]{images/producciones/contraccion.png}
	        \caption[Cambio de tamaño debido a la contraccion del material.]{Cambio de tamaño debido a la contraccion del material. Fuente \cite{tecno_polimeros}}
	        \label{fig:prod_contraccion}
		\end{figure}
	}
\end{itemize}

\subsection{Resultados}
Vamos a comprobar el funcionamiento del sistema con el control de regulación de la propia filawinder, en la que intenta mantener el filamento en una determinada altura y en función de esa altura, aplicar más o menos velocidad de bobinado. Para esta primera producción se usarán todos los pellets reciclados como materia prima, no se mezclará con PLA transparente, para así ver si el funcionamiento de la filastruder es el correcto.\\

Una vez llegado a la consigna de temperatura de 150ºC se comienza a extruir PLA reciclado y se almacena en la bobina. A simple vista, se ve que la velocidad mínima de bobinado hace que el filamento salga demasiado delgado y se generan demasiadas tensiones en el bobinado del mismo. No obstante, analizaremos los resultados obtenidos. Para el análisis del fihero CSV generado, se usará la herramienta IPython con las librerias, Numpy y Scypi, con los que de manera rápida obtendremos unas conclusiones.\\

Las condiciones iniciales del ensayo son:

	\begin{itemize}
		\item Hora de inicio: 11:50
		\item Hora de fin: 12:20
		\item Temperatura de extrusión: 150ºC
	\end{itemize}

Las medidas obtenidas en el ensayo son:

\begin{table}[H]
	\centering
	\begin{tabular}{ccc}
		{\bf } & {\bf Diámetro X} & {\bf Diámetro Y} \\ \hline
		Medidas & 1110.000000 & 1110.000000 \\
		Media (mm) & 1.17 & 0.91 \\
		Desviación estandar & 0.39 & 0.51 \\
		Mínimo (mm) & 0.01 & 0.00 \\
		Máximo (mm) & 1.92 & 1.74
	\end{tabular}
	\caption{Resultados obtenidos en la producción}
	\label{tab:result1}
\end{table}

Cómo podemos comprobar obtenemos una media aritmética de filamento de $ \bar{x} = 1.17mm $  mm con una desviación estandar $\sigma = 0.39$. Representando las medidas de los ejes X e Y podemos comprobar que el resultado no es el deseado:

\begin{figure}[H]
	\centering
	\includegraphics[width=0.8\textwidth]{images/producciones/16062015/output_9_1.png}
	\caption{Representación de las medidas de los ejes X e Y.}
	\label{fig:prod_ejes}
\end{figure}

Sólo un pequeño número de muestras están por dentro de los márgenes de calidad establecidos por BQ (Max = 1.85mm ; Min = 1.65mm). Además, si representamos los datos en un diagráma de cajas, vemos que hay mucha variación entre los dos ejes:
\begin{figure}[H]
    \centering
    \includegraphics[width=0.5\textwidth]{images/producciones/16062015/output_10_1.png}
    \caption{Diagrama de cajas de las medidas de los ejes X e Y.}
    \label{fig:prod_boxplot}
\end{figure}

Después de los resultados obtenidos en el experimento, descartamos que con este esquema de producción y los materiales disponibles, lleguemos a regular el diámetro de salida de una forma óptima, por tanto se trata de obtener otro esquema de producción para ganarantizar mejores resultados.

\section{Filastruder-sensor diámetro-Tractora}
\label{sec:FST}

En una de las pruebas de extrusión de filamento, se va estirando a mano según va saliendo de la filastruder y se comprueba que, traccionando del hilo a una distancia cercana de la boquilla y una vez enfriado, se puede llegar a regular el diámetro final. Por ello, se trata de investigar en esta linea, diseñar un sistema capaz de traccionar el filamento a medida que va saliendo de la boquilla.

\begin{figure}[H]
    \centering
    \includegraphics[width=0.6\textwidth]{images/producciones/Diagram2.png}
    \caption{Esquema de producción filastruder-sensor de diámetro-tractora.}
    \label{fig:esquemap_FST}
\end{figure}

Con el conocimiento adquirido al diseñar la peletizadora (ver anexo \ref{ane:peletizadora}) se diseña una unidad tractora que irá colocada después de la filastruder, la cual, deberá ir traccionando del filamento independientemente del diámetro del mismo. Así mismo, deberemos ser capaces de regular la velocidad de una forma más precisa que como lo hicimos con la peletizadora. En el anexo \ref{ane:tractora} se detalla el proceso seguido a la hora del diseño de la tractora.\\

Por tanto, podemos pasar a instalar la tractora en la maqueta.

\begin{figure}[H]
    \centering
    \includegraphics[width=0.6\textwidth]{images/producciones/tractora/IMG_20150709_130326.jpg}
    \caption{Montaje final filastruder-tractora-sensor}
    \label{fig:montaje_final}
\end{figure}

\subsection{Resultados}

Usando pellets reciclados, se pasa a hacer una producción de filamento, para comprobar que el diseño de la tractora es el correcto. El ensayo que se va a realizar consiste en:

\begin{itemize}
    \item{Establecer una temperatura de 135ºC en el extrusor.}
    \item{Llenar la tolva que incluye de serie la extrusora con 42gr de pellets reciclados.}
    \item{Extruir filamento registrando los datos para su posterior análisis.}
    \item{Cambiar la velocidad de tracción para comprobar la relación final en el diámetro. Se establecera una velocidad de 1RPM y 3RPM}
\end{itemize}

Se está alrededor de seis minutos extruyendo filamento, sin embargo, debido a un exceso en el diámetro del filamento, hace que este no entre por el sensor de diámetro y sea necesario parar, sin embargo podemos estudiar los datos obtenidos.
Tras el ensayo, los resultados obtenidos son los siguientes:

\begin{table}[H]
    \centering
    \begin{tabular}{cc}
               & Diámetro X \\ \hline
    Medidas    & 203        \\
    Media (mm) & 1.59       \\
    Desviación estandar & 0.25\\
    Mínimo (mm)   & 1.08       \\
    Máximo (mm)   & 2.19      
    \end{tabular}
    \caption{Datos obtenidos en el ensayo del día 20 de Julio}
    \label{tab:20007105-dat}
\end{table}

\begin{figure}[H]
    \centering
    \includegraphics[width=0.99\textwidth]{images/producciones/20072015/graficas.png}
    \caption{Gráfica con los datos de la producción}
    \label{fig:2007105-graf}
\end{figure}

En la gráfica anterior vemos los datos obtenidos del diámetro del eje X. Se ve claramente, que hay una variación muy pronunciada al tener una velocidad de tracción constante, esto es un problema en el diseño de la filastruder ya que, a una velocidad de tracción constante, el diámetro debería serlo también. Durante el ensayo, se ha notado física y acústicamente que la extrusión no es constante, por ello, pensamos que el funcionamiento de la filastruder, que a simple vista parecía correcto, no puede llegar a proporcionarnos los resultados que necesitamos. Se instala un encoder mediante un imán, un sensor de efecto hall y un arduino a la filastruder, para cercionarnos de que la velocidad es constante.

\begin{figure}[H]
    \centering
    \includegraphics[width=0.8\textwidth]{images/producciones/20072015/IMG_20150721_110502.jpg}
    \caption{Encoder instalado en la filastruder.}
    \label{fig:2007105-enc}
\end{figure}

Se extruye una cantidad de filamento sin medir el diámetro, tan sólo registrando la velocidad con la que gira el husillo y los datos proporcionados son los siguientes:

\begin{figure}[H]
    \centering
    \includegraphics[width=0.99\textwidth]{images/producciones/20072015/RPM_tract.png}
    \caption{Velocidad de la extrusora.}
    \label{fig:2007105-grafenc}
\end{figure}

Se ve claramente, que la velocidad de giro no es constante, variando entre 1 y 2 RPM. El motor que hace girar el husillo está conectado directamente a 12V y no se tienen ningún control sobre él. Además mecánicamente, el motor está conectado al husillo mediante una caja reductora, la cual no podemos cambiar. Para no alargar la duración del proyecto e intentar avanzar en el control se van a tomar las siguientes medidas:

\begin{itemize}
    \item{Usar una mezcla de granza de PLA natural (70\%) con pellets reciclados de filamento (30\%) (ver figura \ref{fig:2007105-mezc}). Se usará granza de PLA natural, el cual tiene una forma más redondeada que los pellets reciclados, haciendo más fácil el avance dentro del cañón. Se usa una mezcla con los pellets reciclados, ya que el sensor de diámetro que usamos, no funciona con filamento transparente, por ello, debemos tintar el filamento}
	    \begin{figure}[H]
		    \centering
		    \includegraphics[width=0.6\textwidth]{images/producciones/20072015/IMG_20150903_155859.jpg}
		    \caption{Mezcla de granza con pellets reciclados.}
		    \label{fig:2007105-mezc}
		\end{figure}
    \item{Antes de hacer una producción de filamento, la granza de PLA que se vaya a usar, se va a secar en un horno a una temperatura de 80ºC durante, al menos, tres horas antes de la producción. Esto es debido a que si el PLA tiene un alto porcentaje de humedad, hará que la extrusión del material no sea el adecuado, afectando directamente en el acabado final.}
    \item{Se va a diseñar una tolva de alimentación mayor, para que la capacidad de granza aumente, y se ejerza mayor presión a la entrada del extrusor, para que de ese modo, la alimentación de la granza se lo más constante posible. El tamaño de almacenamiento máximo ha pasado de 42gr a 150gr (ver figura \ref{fig:tolv_montaj}).}
\end{itemize}

\begin{figure}[H]
    \centering
    \begin{subfigure}[b]{0.45\textwidth}
        \centering
        \includegraphics[width=\linewidth]{images/producciones/20072015/IMG_20150721_121831.jpg}
        \label{fig:tolva-impresa}
    \end{subfigure}
    ~
    \begin{subfigure}[b]{0.45\textwidth}
            \centering
        \includegraphics[width=\linewidth]{images/producciones/20072015/IMG_20150721_121904.jpg}
        \label{fig:tolva-montada}
    \end{subfigure}
    \caption[Diseño y montaje de una tolva de mayor capacidad.]{Diseño y montaje de una tolva de mayor capacidad.A la izquierda, vemos la tolva recien impresa. A la derecha, la tolva montada en la filastruder.}
    \label{fig:tolv_montaj}
\end{figure}

Se muestran a continuación los resultados obtenidos en el ensayo:

\begin{table}[H]
    \centering
    \begin{tabular}{ccc}
                            & Tolva Grande & Tolva pequeña \\ \hline
        Medidas               & 2000.000000  & 2000.000000   \\
        Media (mm)          & 1.63     & 1.55      \\
        Desviación estandar & 0.14     & 0.15      \\
        Mínimo (mm)             & 1.02     & 0.01      \\
        Máximo (mm)             & 2.20     & 2.40     
    \end{tabular}
    \caption{Datos del ensayo con distintas tolvas}
    \label{tab:ensa_tolvas}
\end{table}

\begin{figure}[H]
    \centering
    \includegraphics[width=0.8\textwidth]{images/producciones/22072015/output_6_1.png}
    \caption{Diagrama de cajas }
    \label{fig:22072015-boxplot}
\end{figure}

Como se ve en el diagrama de cajas, en el que se representa la distribución de los datos, los datos obtenidos en el ensayo con la tolva grande, son algo más estables, por lo tanto, podemos confirmar, que las medidas tomadas anteriormente para intentar mejorar la producción son acertadas, sin embargo, como se aprecia en el gráfico siguiente, se sigue teniendo una variación muy alta en el sistema, lo cual es un problema para intentar integrar un regulador del tipo PID. Por ello, se decide implementar un regulador experto para intentar controlar de forma más precisa el diámetro.


