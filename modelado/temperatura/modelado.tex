
% Default to the notebook output style

    


% Inherit from the specified cell style.




    
\documentclass{article}

    
    
    \usepackage{graphicx} % Used to insert images
    \usepackage{adjustbox} % Used to constrain images to a maximum size 
    \usepackage{color} % Allow colors to be defined
    \usepackage{enumerate} % Needed for markdown enumerations to work
    \usepackage{geometry} % Used to adjust the document margins
    \usepackage{amsmath} % Equations
    \usepackage{amssymb} % Equations
    \usepackage{eurosym} % defines \euro
    \usepackage[mathletters]{ucs} % Extended unicode (utf-8) support
    \usepackage[utf8x]{inputenc} % Allow utf-8 characters in the tex document
    \usepackage{fancyvrb} % verbatim replacement that allows latex
    \usepackage{grffile} % extends the file name processing of package graphics 
                         % to support a larger range 
    % The hyperref package gives us a pdf with properly built
    % internal navigation ('pdf bookmarks' for the table of contents,
    % internal cross-reference links, web links for URLs, etc.)
    \usepackage{hyperref}
    \usepackage{longtable} % longtable support required by pandoc >1.10
    \usepackage{booktabs}  % table support for pandoc > 1.12.2
    

    
    
    \definecolor{orange}{cmyk}{0,0.4,0.8,0.2}
    \definecolor{darkorange}{rgb}{.71,0.21,0.01}
    \definecolor{darkgreen}{rgb}{.12,.54,.11}
    \definecolor{myteal}{rgb}{.26, .44, .56}
    \definecolor{gray}{gray}{0.45}
    \definecolor{lightgray}{gray}{.95}
    \definecolor{mediumgray}{gray}{.8}
    \definecolor{inputbackground}{rgb}{.95, .95, .85}
    \definecolor{outputbackground}{rgb}{.95, .95, .95}
    \definecolor{traceback}{rgb}{1, .95, .95}
    % ansi colors
    \definecolor{red}{rgb}{.6,0,0}
    \definecolor{green}{rgb}{0,.65,0}
    \definecolor{brown}{rgb}{0.6,0.6,0}
    \definecolor{blue}{rgb}{0,.145,.698}
    \definecolor{purple}{rgb}{.698,.145,.698}
    \definecolor{cyan}{rgb}{0,.698,.698}
    \definecolor{lightgray}{gray}{0.5}
    
    % bright ansi colors
    \definecolor{darkgray}{gray}{0.25}
    \definecolor{lightred}{rgb}{1.0,0.39,0.28}
    \definecolor{lightgreen}{rgb}{0.48,0.99,0.0}
    \definecolor{lightblue}{rgb}{0.53,0.81,0.92}
    \definecolor{lightpurple}{rgb}{0.87,0.63,0.87}
    \definecolor{lightcyan}{rgb}{0.5,1.0,0.83}
    
    % commands and environments needed by pandoc snippets
    % extracted from the output of `pandoc -s`
    \providecommand{\tightlist}{%
      \setlength{\itemsep}{0pt}\setlength{\parskip}{0pt}}
    \DefineVerbatimEnvironment{Highlighting}{Verbatim}{commandchars=\\\{\}}
    % Add ',fontsize=\small' for more characters per line
    \newenvironment{Shaded}{}{}
    \newcommand{\KeywordTok}[1]{\textcolor[rgb]{0.00,0.44,0.13}{\textbf{{#1}}}}
    \newcommand{\DataTypeTok}[1]{\textcolor[rgb]{0.56,0.13,0.00}{{#1}}}
    \newcommand{\DecValTok}[1]{\textcolor[rgb]{0.25,0.63,0.44}{{#1}}}
    \newcommand{\BaseNTok}[1]{\textcolor[rgb]{0.25,0.63,0.44}{{#1}}}
    \newcommand{\FloatTok}[1]{\textcolor[rgb]{0.25,0.63,0.44}{{#1}}}
    \newcommand{\CharTok}[1]{\textcolor[rgb]{0.25,0.44,0.63}{{#1}}}
    \newcommand{\StringTok}[1]{\textcolor[rgb]{0.25,0.44,0.63}{{#1}}}
    \newcommand{\CommentTok}[1]{\textcolor[rgb]{0.38,0.63,0.69}{\textit{{#1}}}}
    \newcommand{\OtherTok}[1]{\textcolor[rgb]{0.00,0.44,0.13}{{#1}}}
    \newcommand{\AlertTok}[1]{\textcolor[rgb]{1.00,0.00,0.00}{\textbf{{#1}}}}
    \newcommand{\FunctionTok}[1]{\textcolor[rgb]{0.02,0.16,0.49}{{#1}}}
    \newcommand{\RegionMarkerTok}[1]{{#1}}
    \newcommand{\ErrorTok}[1]{\textcolor[rgb]{1.00,0.00,0.00}{\textbf{{#1}}}}
    \newcommand{\NormalTok}[1]{{#1}}
    
    % Define a nice break command that doesn't care if a line doesn't already
    % exist.
    \def\br{\hspace*{\fill} \\* }
    % Math Jax compatability definitions
    \def\gt{>}
    \def\lt{<}
    % Document parameters
    \title{modelado}
    
    
    

    % Pygments definitions
    
\makeatletter
\def\PY@reset{\let\PY@it=\relax \let\PY@bf=\relax%
    \let\PY@ul=\relax \let\PY@tc=\relax%
    \let\PY@bc=\relax \let\PY@ff=\relax}
\def\PY@tok#1{\csname PY@tok@#1\endcsname}
\def\PY@toks#1+{\ifx\relax#1\empty\else%
    \PY@tok{#1}\expandafter\PY@toks\fi}
\def\PY@do#1{\PY@bc{\PY@tc{\PY@ul{%
    \PY@it{\PY@bf{\PY@ff{#1}}}}}}}
\def\PY#1#2{\PY@reset\PY@toks#1+\relax+\PY@do{#2}}

\expandafter\def\csname PY@tok@gr\endcsname{\def\PY@tc##1{\textcolor[rgb]{1.00,0.00,0.00}{##1}}}
\expandafter\def\csname PY@tok@cm\endcsname{\let\PY@it=\textit\def\PY@tc##1{\textcolor[rgb]{0.25,0.50,0.50}{##1}}}
\expandafter\def\csname PY@tok@m\endcsname{\def\PY@tc##1{\textcolor[rgb]{0.40,0.40,0.40}{##1}}}
\expandafter\def\csname PY@tok@nd\endcsname{\def\PY@tc##1{\textcolor[rgb]{0.67,0.13,1.00}{##1}}}
\expandafter\def\csname PY@tok@gt\endcsname{\def\PY@tc##1{\textcolor[rgb]{0.00,0.27,0.87}{##1}}}
\expandafter\def\csname PY@tok@nl\endcsname{\def\PY@tc##1{\textcolor[rgb]{0.63,0.63,0.00}{##1}}}
\expandafter\def\csname PY@tok@kn\endcsname{\let\PY@bf=\textbf\def\PY@tc##1{\textcolor[rgb]{0.00,0.50,0.00}{##1}}}
\expandafter\def\csname PY@tok@vi\endcsname{\def\PY@tc##1{\textcolor[rgb]{0.10,0.09,0.49}{##1}}}
\expandafter\def\csname PY@tok@sr\endcsname{\def\PY@tc##1{\textcolor[rgb]{0.73,0.40,0.53}{##1}}}
\expandafter\def\csname PY@tok@ow\endcsname{\let\PY@bf=\textbf\def\PY@tc##1{\textcolor[rgb]{0.67,0.13,1.00}{##1}}}
\expandafter\def\csname PY@tok@vg\endcsname{\def\PY@tc##1{\textcolor[rgb]{0.10,0.09,0.49}{##1}}}
\expandafter\def\csname PY@tok@nf\endcsname{\def\PY@tc##1{\textcolor[rgb]{0.00,0.00,1.00}{##1}}}
\expandafter\def\csname PY@tok@s2\endcsname{\def\PY@tc##1{\textcolor[rgb]{0.73,0.13,0.13}{##1}}}
\expandafter\def\csname PY@tok@sd\endcsname{\let\PY@it=\textit\def\PY@tc##1{\textcolor[rgb]{0.73,0.13,0.13}{##1}}}
\expandafter\def\csname PY@tok@err\endcsname{\def\PY@bc##1{\setlength{\fboxsep}{0pt}\fcolorbox[rgb]{1.00,0.00,0.00}{1,1,1}{\strut ##1}}}
\expandafter\def\csname PY@tok@k\endcsname{\let\PY@bf=\textbf\def\PY@tc##1{\textcolor[rgb]{0.00,0.50,0.00}{##1}}}
\expandafter\def\csname PY@tok@gs\endcsname{\let\PY@bf=\textbf}
\expandafter\def\csname PY@tok@w\endcsname{\def\PY@tc##1{\textcolor[rgb]{0.73,0.73,0.73}{##1}}}
\expandafter\def\csname PY@tok@sc\endcsname{\def\PY@tc##1{\textcolor[rgb]{0.73,0.13,0.13}{##1}}}
\expandafter\def\csname PY@tok@cp\endcsname{\def\PY@tc##1{\textcolor[rgb]{0.74,0.48,0.00}{##1}}}
\expandafter\def\csname PY@tok@ss\endcsname{\def\PY@tc##1{\textcolor[rgb]{0.10,0.09,0.49}{##1}}}
\expandafter\def\csname PY@tok@nb\endcsname{\def\PY@tc##1{\textcolor[rgb]{0.00,0.50,0.00}{##1}}}
\expandafter\def\csname PY@tok@na\endcsname{\def\PY@tc##1{\textcolor[rgb]{0.49,0.56,0.16}{##1}}}
\expandafter\def\csname PY@tok@kd\endcsname{\let\PY@bf=\textbf\def\PY@tc##1{\textcolor[rgb]{0.00,0.50,0.00}{##1}}}
\expandafter\def\csname PY@tok@nn\endcsname{\let\PY@bf=\textbf\def\PY@tc##1{\textcolor[rgb]{0.00,0.00,1.00}{##1}}}
\expandafter\def\csname PY@tok@no\endcsname{\def\PY@tc##1{\textcolor[rgb]{0.53,0.00,0.00}{##1}}}
\expandafter\def\csname PY@tok@se\endcsname{\let\PY@bf=\textbf\def\PY@tc##1{\textcolor[rgb]{0.73,0.40,0.13}{##1}}}
\expandafter\def\csname PY@tok@mh\endcsname{\def\PY@tc##1{\textcolor[rgb]{0.40,0.40,0.40}{##1}}}
\expandafter\def\csname PY@tok@il\endcsname{\def\PY@tc##1{\textcolor[rgb]{0.40,0.40,0.40}{##1}}}
\expandafter\def\csname PY@tok@mi\endcsname{\def\PY@tc##1{\textcolor[rgb]{0.40,0.40,0.40}{##1}}}
\expandafter\def\csname PY@tok@ge\endcsname{\let\PY@it=\textit}
\expandafter\def\csname PY@tok@o\endcsname{\def\PY@tc##1{\textcolor[rgb]{0.40,0.40,0.40}{##1}}}
\expandafter\def\csname PY@tok@sb\endcsname{\def\PY@tc##1{\textcolor[rgb]{0.73,0.13,0.13}{##1}}}
\expandafter\def\csname PY@tok@gi\endcsname{\def\PY@tc##1{\textcolor[rgb]{0.00,0.63,0.00}{##1}}}
\expandafter\def\csname PY@tok@c1\endcsname{\let\PY@it=\textit\def\PY@tc##1{\textcolor[rgb]{0.25,0.50,0.50}{##1}}}
\expandafter\def\csname PY@tok@kr\endcsname{\let\PY@bf=\textbf\def\PY@tc##1{\textcolor[rgb]{0.00,0.50,0.00}{##1}}}
\expandafter\def\csname PY@tok@ne\endcsname{\let\PY@bf=\textbf\def\PY@tc##1{\textcolor[rgb]{0.82,0.25,0.23}{##1}}}
\expandafter\def\csname PY@tok@gu\endcsname{\let\PY@bf=\textbf\def\PY@tc##1{\textcolor[rgb]{0.50,0.00,0.50}{##1}}}
\expandafter\def\csname PY@tok@nc\endcsname{\let\PY@bf=\textbf\def\PY@tc##1{\textcolor[rgb]{0.00,0.00,1.00}{##1}}}
\expandafter\def\csname PY@tok@cs\endcsname{\let\PY@it=\textit\def\PY@tc##1{\textcolor[rgb]{0.25,0.50,0.50}{##1}}}
\expandafter\def\csname PY@tok@mo\endcsname{\def\PY@tc##1{\textcolor[rgb]{0.40,0.40,0.40}{##1}}}
\expandafter\def\csname PY@tok@s1\endcsname{\def\PY@tc##1{\textcolor[rgb]{0.73,0.13,0.13}{##1}}}
\expandafter\def\csname PY@tok@gd\endcsname{\def\PY@tc##1{\textcolor[rgb]{0.63,0.00,0.00}{##1}}}
\expandafter\def\csname PY@tok@kp\endcsname{\def\PY@tc##1{\textcolor[rgb]{0.00,0.50,0.00}{##1}}}
\expandafter\def\csname PY@tok@s\endcsname{\def\PY@tc##1{\textcolor[rgb]{0.73,0.13,0.13}{##1}}}
\expandafter\def\csname PY@tok@gh\endcsname{\let\PY@bf=\textbf\def\PY@tc##1{\textcolor[rgb]{0.00,0.00,0.50}{##1}}}
\expandafter\def\csname PY@tok@mf\endcsname{\def\PY@tc##1{\textcolor[rgb]{0.40,0.40,0.40}{##1}}}
\expandafter\def\csname PY@tok@mb\endcsname{\def\PY@tc##1{\textcolor[rgb]{0.40,0.40,0.40}{##1}}}
\expandafter\def\csname PY@tok@bp\endcsname{\def\PY@tc##1{\textcolor[rgb]{0.00,0.50,0.00}{##1}}}
\expandafter\def\csname PY@tok@nv\endcsname{\def\PY@tc##1{\textcolor[rgb]{0.10,0.09,0.49}{##1}}}
\expandafter\def\csname PY@tok@go\endcsname{\def\PY@tc##1{\textcolor[rgb]{0.53,0.53,0.53}{##1}}}
\expandafter\def\csname PY@tok@kt\endcsname{\def\PY@tc##1{\textcolor[rgb]{0.69,0.00,0.25}{##1}}}
\expandafter\def\csname PY@tok@vc\endcsname{\def\PY@tc##1{\textcolor[rgb]{0.10,0.09,0.49}{##1}}}
\expandafter\def\csname PY@tok@sh\endcsname{\def\PY@tc##1{\textcolor[rgb]{0.73,0.13,0.13}{##1}}}
\expandafter\def\csname PY@tok@kc\endcsname{\let\PY@bf=\textbf\def\PY@tc##1{\textcolor[rgb]{0.00,0.50,0.00}{##1}}}
\expandafter\def\csname PY@tok@si\endcsname{\let\PY@bf=\textbf\def\PY@tc##1{\textcolor[rgb]{0.73,0.40,0.53}{##1}}}
\expandafter\def\csname PY@tok@ni\endcsname{\let\PY@bf=\textbf\def\PY@tc##1{\textcolor[rgb]{0.60,0.60,0.60}{##1}}}
\expandafter\def\csname PY@tok@nt\endcsname{\let\PY@bf=\textbf\def\PY@tc##1{\textcolor[rgb]{0.00,0.50,0.00}{##1}}}
\expandafter\def\csname PY@tok@sx\endcsname{\def\PY@tc##1{\textcolor[rgb]{0.00,0.50,0.00}{##1}}}
\expandafter\def\csname PY@tok@gp\endcsname{\let\PY@bf=\textbf\def\PY@tc##1{\textcolor[rgb]{0.00,0.00,0.50}{##1}}}
\expandafter\def\csname PY@tok@c\endcsname{\let\PY@it=\textit\def\PY@tc##1{\textcolor[rgb]{0.25,0.50,0.50}{##1}}}

\def\PYZbs{\char`\\}
\def\PYZus{\char`\_}
\def\PYZob{\char`\{}
\def\PYZcb{\char`\}}
\def\PYZca{\char`\^}
\def\PYZam{\char`\&}
\def\PYZlt{\char`\<}
\def\PYZgt{\char`\>}
\def\PYZsh{\char`\#}
\def\PYZpc{\char`\%}
\def\PYZdl{\char`\$}
\def\PYZhy{\char`\-}
\def\PYZsq{\char`\'}
\def\PYZdq{\char`\"}
\def\PYZti{\char`\~}
% for compatibility with earlier versions
\def\PYZat{@}
\def\PYZlb{[}
\def\PYZrb{]}
\makeatother


    % Exact colors from NB
    \definecolor{incolor}{rgb}{0.0, 0.0, 0.5}
    \definecolor{outcolor}{rgb}{0.545, 0.0, 0.0}



    
    % Prevent overflowing lines due to hard-to-break entities
    \sloppy 
    % Setup hyperref package
    \hypersetup{
      breaklinks=true,  % so long urls are correctly broken across lines
      colorlinks=true,
      urlcolor=blue,
      linkcolor=darkorange,
      citecolor=darkgreen,
      }
    % Slightly bigger margins than the latex defaults
    
    \geometry{verbose,tmargin=1in,bmargin=1in,lmargin=1in,rmargin=1in}
    
    

    \begin{document}
    
    
    \maketitle
    
    

    
    \section{Modelado de un sistema con
ipython}\label{modelado-de-un-sistema-con-ipython}

    Uso de ipython para el modelado de un sistema a partir de los datos
obtenidos en un ensayo.

    \begin{Verbatim}[commandchars=\\\{\}]
{\color{incolor}In [{\color{incolor}47}]:} \PY{c}{\PYZsh{}Importamos las librerías utilizadas}
         \PY{k+kn}{import} \PY{n+nn}{numpy} \PY{k}{as} \PY{n+nn}{np}
         \PY{k+kn}{import} \PY{n+nn}{pandas} \PY{k}{as} \PY{n+nn}{pd}
         \PY{k+kn}{import} \PY{n+nn}{seaborn} \PY{k}{as} \PY{n+nn}{sns}
         \PY{k+kn}{from} \PY{n+nn}{scipy} \PY{k}{import} \PY{n}{signal}
\end{Verbatim}

    \begin{Verbatim}[commandchars=\\\{\}]
{\color{incolor}In [{\color{incolor}48}]:} \PY{c}{\PYZsh{}Mostramos las versiones usadas de cada librerías}
         \PY{n+nb}{print} \PY{p}{(}\PY{l+s}{\PYZdq{}}\PY{l+s}{Numpy v\PYZob{}\PYZcb{}}\PY{l+s}{\PYZdq{}}\PY{o}{.}\PY{n}{format}\PY{p}{(}\PY{n}{np}\PY{o}{.}\PY{n}{\PYZus{}\PYZus{}version\PYZus{}\PYZus{}}\PY{p}{)}\PY{p}{)}
         \PY{n+nb}{print} \PY{p}{(}\PY{l+s}{\PYZdq{}}\PY{l+s}{Pandas v\PYZob{}\PYZcb{}}\PY{l+s}{\PYZdq{}}\PY{o}{.}\PY{n}{format}\PY{p}{(}\PY{n}{pd}\PY{o}{.}\PY{n}{\PYZus{}\PYZus{}version\PYZus{}\PYZus{}}\PY{p}{)}\PY{p}{)}
         \PY{n+nb}{print} \PY{p}{(}\PY{l+s}{\PYZdq{}}\PY{l+s}{Seaborn v\PYZob{}\PYZcb{}}\PY{l+s}{\PYZdq{}}\PY{o}{.}\PY{n}{format}\PY{p}{(}\PY{n}{sns}\PY{o}{.}\PY{n}{\PYZus{}\PYZus{}version\PYZus{}\PYZus{}}\PY{p}{)}\PY{p}{)}
\end{Verbatim}

    \begin{Verbatim}[commandchars=\\\{\}]
Numpy v1.9.2
Pandas v0.16.2
Seaborn v0.6.0
    \end{Verbatim}

    \begin{Verbatim}[commandchars=\\\{\}]
{\color{incolor}In [{\color{incolor}49}]:} \PY{o}{\PYZpc{}}\PY{k}{pylab} inline
\end{Verbatim}

    \begin{Verbatim}[commandchars=\\\{\}]
Populating the interactive namespace from numpy and matplotlib
    \end{Verbatim}

    \begin{Verbatim}[commandchars=\\\{\}]
{\color{incolor}In [{\color{incolor}50}]:} \PY{c}{\PYZsh{}Abrimos el fichero csv con los datos de la muestra}
         \PY{n}{datos} \PY{o}{=} \PY{n}{pd}\PY{o}{.}\PY{n}{read\PYZus{}csv}\PY{p}{(}\PY{l+s}{\PYZsq{}}\PY{l+s}{datos.csv}\PY{l+s}{\PYZsq{}}\PY{p}{)}
\end{Verbatim}

    \begin{Verbatim}[commandchars=\\\{\}]
{\color{incolor}In [{\color{incolor}51}]:} \PY{c}{\PYZsh{}Almacenamos en una lista las columnas del fichero con las que vamos a trabajar}
         \PY{c}{\PYZsh{}columns = [\PYZsq{}temperatura\PYZsq{}, \PYZsq{}entrada\PYZsq{}]}
         \PY{n}{columns} \PY{o}{=} \PY{p}{[}\PY{l+s}{\PYZsq{}}\PY{l+s}{temperatura}\PY{l+s}{\PYZsq{}}\PY{p}{,} \PY{l+s}{\PYZsq{}}\PY{l+s}{entrada}\PY{l+s}{\PYZsq{}}\PY{p}{]}
\end{Verbatim}

    \subsection{Representación}\label{representaciuxf3n}

    Representamos los datos mostrados en función del tiempo. De esta manera,
vemos la respuesta física que tiene nuestro sistema.

    \begin{Verbatim}[commandchars=\\\{\}]
{\color{incolor}In [{\color{incolor}52}]:} \PY{c}{\PYZsh{}Mostramos en varias gráficas la información obtenida tras el ensayo}
         \PY{n}{ax} \PY{o}{=} \PY{n}{datos}\PY{p}{[}\PY{n}{columns}\PY{p}{]}\PY{o}{.}\PY{n}{plot}\PY{p}{(}\PY{n}{secondary\PYZus{}y}\PY{o}{=}\PY{p}{[}\PY{l+s}{\PYZsq{}}\PY{l+s}{entrada}\PY{l+s}{\PYZsq{}}\PY{p}{]}\PY{p}{,}\PY{n}{figsize}\PY{o}{=}\PY{p}{(}\PY{l+m+mi}{10}\PY{p}{,}\PY{l+m+mi}{5}\PY{p}{)}\PY{p}{,} \PY{n}{ylim}\PY{o}{=}\PY{p}{(}\PY{l+m+mi}{20}\PY{p}{,}\PY{l+m+mi}{60}\PY{p}{)}\PY{p}{,}\PY{n}{title}\PY{o}{=}\PY{l+s}{\PYZsq{}}\PY{l+s}{Modelo matemático del sistema}\PY{l+s}{\PYZsq{}}\PY{p}{)}
         \PY{n}{ax}\PY{o}{.}\PY{n}{set\PYZus{}xlabel}\PY{p}{(}\PY{l+s}{\PYZsq{}}\PY{l+s}{Tiempo}\PY{l+s}{\PYZsq{}}\PY{p}{)}
         \PY{n}{ax}\PY{o}{.}\PY{n}{set\PYZus{}ylabel}\PY{p}{(}\PY{l+s}{\PYZsq{}}\PY{l+s}{Temperatura [ºC]}\PY{l+s}{\PYZsq{}}\PY{p}{)}
         \PY{c}{\PYZsh{}datos\PYZus{}filtrados[\PYZsq{}RPM TRAC\PYZsq{}].plot(secondary\PYZus{}y=True,style=\PYZsq{}g\PYZsq{},figsize=(20,20)).set\PYZus{}ylabel=(\PYZsq{}RPM\PYZsq{})}
\end{Verbatim}

            \begin{Verbatim}[commandchars=\\\{\}]
{\color{outcolor}Out[{\color{outcolor}52}]:} <matplotlib.text.Text at 0x9e69c70>
\end{Verbatim}
        
    \begin{center}
    \adjustimage{max size={0.9\linewidth}{0.9\paperheight}}{modelado_files/modelado_9_1.png}
    \end{center}
    { \hspace*{\fill} \\}
    
    \subsection{Cálculo del polinomio}\label{cuxe1lculo-del-polinomio}

    Hacemos una regresión con un polinomio de orden 4 para calcular cual es
la mejor ecuación que se ajusta a la tendencia de nuestros datos.

    \begin{Verbatim}[commandchars=\\\{\}]
{\color{incolor}In [{\color{incolor}53}]:} \PY{c}{\PYZsh{} Buscamos el polinomio de orden 4 que determina la distribución de los datos}
         \PY{n}{reg} \PY{o}{=} \PY{n}{np}\PY{o}{.}\PY{n}{polyfit}\PY{p}{(}\PY{n}{datos}\PY{p}{[}\PY{l+s}{\PYZsq{}}\PY{l+s}{time}\PY{l+s}{\PYZsq{}}\PY{p}{]}\PY{p}{,}\PY{n}{datos}\PY{p}{[}\PY{l+s}{\PYZsq{}}\PY{l+s}{temperatura}\PY{l+s}{\PYZsq{}}\PY{p}{]}\PY{p}{,}\PY{l+m+mi}{2}\PY{p}{)}
         \PY{c}{\PYZsh{} Calculamos los valores de y con la regresión}
         \PY{n}{ry} \PY{o}{=} \PY{n}{np}\PY{o}{.}\PY{n}{polyval}\PY{p}{(}\PY{n}{reg}\PY{p}{,}\PY{n}{datos}\PY{p}{[}\PY{l+s}{\PYZsq{}}\PY{l+s}{time}\PY{l+s}{\PYZsq{}}\PY{p}{]}\PY{p}{)}
         \PY{n+nb}{print} \PY{p}{(}\PY{n}{reg}\PY{p}{)}
\end{Verbatim}

    \begin{Verbatim}[commandchars=\\\{\}]
[  8.18174358e-09  -1.57332539e-04   2.59458703e+01]
    \end{Verbatim}

    \begin{Verbatim}[commandchars=\\\{\}]
{\color{incolor}In [{\color{incolor}54}]:} \PY{n}{plt}\PY{o}{.}\PY{n}{plot}\PY{p}{(}\PY{n}{datos}\PY{p}{[}\PY{l+s}{\PYZsq{}}\PY{l+s}{time}\PY{l+s}{\PYZsq{}}\PY{p}{]}\PY{p}{,}\PY{n}{datos}\PY{p}{[}\PY{l+s}{\PYZsq{}}\PY{l+s}{temperatura}\PY{l+s}{\PYZsq{}}\PY{p}{]}\PY{p}{,}\PY{l+s}{\PYZsq{}}\PY{l+s}{b\PYZca{}}\PY{l+s}{\PYZsq{}}\PY{p}{,} \PY{n}{label}\PY{o}{=}\PY{p}{(}\PY{l+s}{\PYZsq{}}\PY{l+s}{Datos experimentales}\PY{l+s}{\PYZsq{}}\PY{p}{)}\PY{p}{)}
         \PY{n}{plt}\PY{o}{.}\PY{n}{plot}\PY{p}{(}\PY{n}{datos}\PY{p}{[}\PY{l+s}{\PYZsq{}}\PY{l+s}{time}\PY{l+s}{\PYZsq{}}\PY{p}{]}\PY{p}{,}\PY{n}{ry}\PY{p}{,}\PY{l+s}{\PYZsq{}}\PY{l+s}{ro}\PY{l+s}{\PYZsq{}}\PY{p}{,} \PY{n}{label}\PY{o}{=}\PY{p}{(}\PY{l+s}{\PYZsq{}}\PY{l+s}{regresión polinómica}\PY{l+s}{\PYZsq{}}\PY{p}{)}\PY{p}{)}
         \PY{n}{plt}\PY{o}{.}\PY{n}{legend}\PY{p}{(}\PY{n}{loc}\PY{o}{=}\PY{l+m+mi}{0}\PY{p}{)}
         \PY{n}{plt}\PY{o}{.}\PY{n}{grid}\PY{p}{(}\PY{k}{True}\PY{p}{)}
         \PY{n}{plt}\PY{o}{.}\PY{n}{xlabel}\PY{p}{(}\PY{l+s}{\PYZsq{}}\PY{l+s}{Tiempo}\PY{l+s}{\PYZsq{}}\PY{p}{)}
         \PY{n}{plt}\PY{o}{.}\PY{n}{ylabel}\PY{p}{(}\PY{l+s}{\PYZsq{}}\PY{l+s}{Temperatura [ºC]}\PY{l+s}{\PYZsq{}}\PY{p}{)}
\end{Verbatim}

            \begin{Verbatim}[commandchars=\\\{\}]
{\color{outcolor}Out[{\color{outcolor}54}]:} <matplotlib.text.Text at 0x9e9d5d0>
\end{Verbatim}
        
    \begin{center}
    \adjustimage{max size={0.9\linewidth}{0.9\paperheight}}{modelado_files/modelado_13_1.png}
    \end{center}
    { \hspace*{\fill} \\}
    
    El polinomio caracteristico de nuestro sistema es:

\[P_x=  25.9459 -1.5733·10^{-4}·X - 8.18174·10^{-9}·X^2\]

    \subsection{Transformada de laplace}\label{transformada-de-laplace}

    Si calculamos la transformada de laplace del sistema, obtenemos el
siguiente resultado:

\[G_s = \frac{25.95·S^2 - 0.00015733·S + 1.63635·10^{-8}}{S^3}\]

    \begin{Verbatim}[commandchars=\\\{\}]
{\color{incolor}In [{\color{incolor}55}]:} \PY{c}{\PYZsh{}\PYZsh{}Respuesta en frecuencia del sistema }
         \PY{n}{num} \PY{o}{=} \PY{p}{[}\PY{l+m+mf}{25.9459} \PY{p}{,}\PY{l+m+mf}{0.00015733} \PY{p}{,}\PY{l+m+mf}{0.00000000818174}\PY{p}{]}
         \PY{n}{den} \PY{o}{=} \PY{p}{[}\PY{l+m+mi}{1}\PY{p}{,}\PY{l+m+mi}{0}\PY{p}{,}\PY{l+m+mi}{0}\PY{p}{]}
         \PY{n}{tf} \PY{o}{=} \PY{n}{signal}\PY{o}{.}\PY{n}{lti}\PY{p}{(}\PY{n}{num}\PY{p}{,}\PY{n}{den}\PY{p}{)}
         \PY{n}{w}\PY{p}{,} \PY{n}{mag}\PY{p}{,} \PY{n}{phase} \PY{o}{=} \PY{n}{signal}\PY{o}{.}\PY{n}{bode}\PY{p}{(}\PY{n}{tf}\PY{p}{)}
\end{Verbatim}

    \begin{Verbatim}[commandchars=\\\{\}]
{\color{incolor}In [{\color{incolor}56}]:} \PY{n}{tf} \PY{o}{=} \PY{n}{signal}\PY{o}{.}\PY{n}{lti}\PY{p}{(}\PY{n}{num}\PY{p}{,}\PY{n}{den}\PY{p}{)}
         \PY{n}{w}\PY{p}{,} \PY{n}{mag}\PY{p}{,} \PY{n}{phase} \PY{o}{=} \PY{n}{signal}\PY{o}{.}\PY{n}{bode}\PY{p}{(}\PY{n}{tf}\PY{p}{)}
\end{Verbatim}

    \begin{Verbatim}[commandchars=\\\{\}]
{\color{incolor}In [{\color{incolor}57}]:} \PY{n}{fig}\PY{p}{,} \PY{p}{(}\PY{n}{ax1}\PY{p}{,} \PY{n}{ax2}\PY{p}{)} \PY{o}{=} \PY{n}{plt}\PY{o}{.}\PY{n}{subplots}\PY{p}{(}\PY{l+m+mi}{2}\PY{p}{,} \PY{l+m+mi}{1}\PY{p}{,} \PY{n}{figsize}\PY{o}{=}\PY{p}{(}\PY{l+m+mi}{6}\PY{p}{,} \PY{l+m+mi}{6}\PY{p}{)}\PY{p}{)}
         \PY{n}{ax1}\PY{o}{.}\PY{n}{semilogx}\PY{p}{(}\PY{n}{w}\PY{p}{,} \PY{n}{mag}\PY{p}{)} \PY{c}{\PYZsh{} Eje x logarítmico}
         \PY{n}{ax2}\PY{o}{.}\PY{n}{semilogx}\PY{p}{(}\PY{n}{w}\PY{p}{,} \PY{n}{phase}\PY{p}{)} \PY{c}{\PYZsh{} Eje x logarítmico}
\end{Verbatim}

            \begin{Verbatim}[commandchars=\\\{\}]
{\color{outcolor}Out[{\color{outcolor}57}]:} [<matplotlib.lines.Line2D at 0x9eb8ef0>]
\end{Verbatim}
        
    \begin{center}
    \adjustimage{max size={0.9\linewidth}{0.9\paperheight}}{modelado_files/modelado_19_1.png}
    \end{center}
    { \hspace*{\fill} \\}
    
    \begin{Verbatim}[commandchars=\\\{\}]
{\color{incolor}In [{\color{incolor}58}]:} \PY{n}{w}\PY{p}{,} \PY{n}{H} \PY{o}{=} \PY{n}{signal}\PY{o}{.}\PY{n}{freqresp}\PY{p}{(}\PY{n}{tf}\PY{p}{)}
         \PY{n}{fig}\PY{p}{,} \PY{p}{(}\PY{n}{ax1}\PY{p}{,} \PY{n}{ax2}\PY{p}{)} \PY{o}{=} \PY{n}{plt}\PY{o}{.}\PY{n}{subplots}\PY{p}{(}\PY{l+m+mi}{1}\PY{p}{,} \PY{l+m+mi}{2}\PY{p}{,} \PY{n}{figsize}\PY{o}{=}\PY{p}{(}\PY{l+m+mi}{10}\PY{p}{,} \PY{l+m+mi}{10}\PY{p}{)}\PY{p}{)}
         \PY{n}{ax1}\PY{o}{.}\PY{n}{plot}\PY{p}{(}\PY{n}{H}\PY{o}{.}\PY{n}{real}\PY{p}{,} \PY{n}{H}\PY{o}{.}\PY{n}{imag}\PY{p}{)}
         \PY{n}{ax1}\PY{o}{.}\PY{n}{plot}\PY{p}{(}\PY{n}{H}\PY{o}{.}\PY{n}{real}\PY{p}{,} \PY{o}{\PYZhy{}}\PY{n}{H}\PY{o}{.}\PY{n}{imag}\PY{p}{)}
         \PY{n}{ax2}\PY{o}{.}\PY{n}{plot}\PY{p}{(}\PY{n}{tf}\PY{o}{.}\PY{n}{zeros}\PY{o}{.}\PY{n}{real}\PY{p}{,} \PY{n}{tf}\PY{o}{.}\PY{n}{zeros}\PY{o}{.}\PY{n}{imag}\PY{p}{,} \PY{l+s}{\PYZsq{}}\PY{l+s}{o}\PY{l+s}{\PYZsq{}}\PY{p}{)}
         \PY{n}{ax2}\PY{o}{.}\PY{n}{plot}\PY{p}{(}\PY{n}{tf}\PY{o}{.}\PY{n}{poles}\PY{o}{.}\PY{n}{real}\PY{p}{,} \PY{n}{tf}\PY{o}{.}\PY{n}{poles}\PY{o}{.}\PY{n}{imag}\PY{p}{,} \PY{l+s}{\PYZsq{}}\PY{l+s}{x}\PY{l+s}{\PYZsq{}}\PY{p}{)}
\end{Verbatim}

            \begin{Verbatim}[commandchars=\\\{\}]
{\color{outcolor}Out[{\color{outcolor}58}]:} [<matplotlib.lines.Line2D at 0xbdb7190>]
\end{Verbatim}
        
    \begin{center}
    \adjustimage{max size={0.9\linewidth}{0.9\paperheight}}{modelado_files/modelado_20_1.png}
    \end{center}
    { \hspace*{\fill} \\}
    
    \begin{Verbatim}[commandchars=\\\{\}]
{\color{incolor}In [{\color{incolor}59}]:} \PY{n}{t}\PY{p}{,} \PY{n}{y} \PY{o}{=} \PY{n}{signal}\PY{o}{.}\PY{n}{step2}\PY{p}{(}\PY{n}{tf}\PY{p}{)} \PY{c}{\PYZsh{} Respuesta a escalón unitario}
         \PY{n}{plt}\PY{o}{.}\PY{n}{plot}\PY{p}{(}\PY{n}{t}\PY{p}{,} \PY{l+m+mi}{2250} \PY{o}{*} \PY{n}{y}\PY{p}{)} \PY{c}{\PYZsh{} Equivalente a una entrada de altura 2250}
\end{Verbatim}

            \begin{Verbatim}[commandchars=\\\{\}]
{\color{outcolor}Out[{\color{outcolor}59}]:} [<matplotlib.lines.Line2D at 0xc005270>]
\end{Verbatim}
        
    \begin{center}
    \adjustimage{max size={0.9\linewidth}{0.9\paperheight}}{modelado_files/modelado_21_1.png}
    \end{center}
    { \hspace*{\fill} \\}
    
    \subsection{Cálculo del PID mediante
OCTAVE}\label{cuxe1lculo-del-pid-mediante-octave}

    Aplicando el método de sintonizacion de Ziegler-Nichols calcularemos el
PID para poder regular correctamente el sistema.Este método, nos da d
emanera rápida unos valores de \(K_p\), \(K_i\) y \(K_d\) orientativos,
para que podamos ajustar correctamente el controlador. Esté método
consiste en el cálculo de tres parámetros característicos, con los
cuales calcularemos el regulador:

\[G_s=K_p(1+\frac{1}{T_i·S}+T_d·s)=K_p+\frac{K_i}{s}+K_d\]

El cálculo de los parámetros característicos del método, lo realizaremos
con Octave, con el siguiente código: \textsubscript{\textasciitilde{}}
pkg load control \%los datos en la función tf() debe ser el numerador y
denominador de nuestro sistema. H=tf({[}25.95 0.000157333
1.63635E-8{]},{[}1 0 0 0{]}); step(H); dt=0.150; t=0:dt:65; y=step(H,t);
dy=diff(y)/dt; {[}m,p{]}=max(dy); yi=y(p); ti=t(p); L=ti-yi/m
Tao=(y(end)-yi)/m+ti-L Kp=1.2\emph{Tao/L Ti=2}L; Td=0.5\emph{L;
Ki=Kp/ti; Kd=Kp}Td; \textsubscript{\textasciitilde{}}

En esta primera iteración, los datos obtenidos son los siguientes:
\(K_p = 6082.6\) \(K_i=93.868 K_d=38.9262\)

Con lo que nuestro regulador tiene la siguiente ecuación característica:

\[G_s = \frac{38.9262·S^2 + 6082.6·S + 93.868}{S}\]

    \subsubsection{Iteracción 1 de
regulador}\label{iteracciuxf3n-1-de-regulador}

    \begin{Verbatim}[commandchars=\\\{\}]
{\color{incolor}In [{\color{incolor}60}]:} \PY{c}{\PYZsh{}Almacenamos en una lista las columnas del fichero con las que vamos a trabajar}
         \PY{n}{datos\PYZus{}it1} \PY{o}{=} \PY{n}{pd}\PY{o}{.}\PY{n}{read\PYZus{}csv}\PY{p}{(}\PY{l+s}{\PYZsq{}}\PY{l+s}{Regulador1.csv}\PY{l+s}{\PYZsq{}}\PY{p}{)}
         \PY{n}{columns} \PY{o}{=} \PY{p}{[}\PY{l+s}{\PYZsq{}}\PY{l+s}{temperatura}\PY{l+s}{\PYZsq{}}\PY{p}{]}
\end{Verbatim}

    \begin{Verbatim}[commandchars=\\\{\}]
{\color{incolor}In [{\color{incolor}61}]:} \PY{c}{\PYZsh{}Mostramos en varias gráficas la información obtenida tras el ensayo}
         \PY{n}{ax} \PY{o}{=} \PY{n}{datos\PYZus{}it1}\PY{p}{[}\PY{n}{columns}\PY{p}{]}\PY{o}{.}\PY{n}{plot}\PY{p}{(}\PY{n}{figsize}\PY{o}{=}\PY{p}{(}\PY{l+m+mi}{10}\PY{p}{,}\PY{l+m+mi}{5}\PY{p}{)}\PY{p}{,} \PY{n}{ylim}\PY{o}{=}\PY{p}{(}\PY{l+m+mi}{20}\PY{p}{,}\PY{l+m+mi}{100}\PY{p}{)}\PY{p}{,}\PY{n}{title}\PY{o}{=}\PY{l+s}{\PYZsq{}}\PY{l+s}{Modelo matemático del sistema con regulador}\PY{l+s}{\PYZsq{}}\PY{p}{,}\PY{p}{)}
         \PY{n}{ax}\PY{o}{.}\PY{n}{set\PYZus{}xlabel}\PY{p}{(}\PY{l+s}{\PYZsq{}}\PY{l+s}{Tiempo}\PY{l+s}{\PYZsq{}}\PY{p}{)}
         \PY{n}{ax}\PY{o}{.}\PY{n}{set\PYZus{}ylabel}\PY{p}{(}\PY{l+s}{\PYZsq{}}\PY{l+s}{Temperatura [ºC]}\PY{l+s}{\PYZsq{}}\PY{p}{)}
\end{Verbatim}

            \begin{Verbatim}[commandchars=\\\{\}]
{\color{outcolor}Out[{\color{outcolor}61}]:} <matplotlib.text.Text at 0xbd88c70>
\end{Verbatim}
        
    \begin{center}
    \adjustimage{max size={0.9\linewidth}{0.9\paperheight}}{modelado_files/modelado_26_1.png}
    \end{center}
    { \hspace*{\fill} \\}
    
    En este caso hemos establecido un setpoint de 80ºC Como vemos, una vez
introducido el controlador, la temperatura tiende a estabilizarse, sin
embargo tiene mucha sobreoscilación. Por ello aumentaremos los valores
de \(K_i\) y \(K_d\), siendo los valores de esta segunda iteracción los
siguientes: \(K_p = 6082.6\) \(K_i=103.25 K_d=51.425\)

    \subsubsection{Iteracción 2 del
regulador}\label{iteracciuxf3n-2-del-regulador}

    \begin{Verbatim}[commandchars=\\\{\}]
{\color{incolor}In [{\color{incolor}68}]:} \PY{c}{\PYZsh{}Almacenamos en una lista las columnas del fichero con las que vamos a trabajar}
         \PY{n}{datos\PYZus{}it2} \PY{o}{=} \PY{n}{pd}\PY{o}{.}\PY{n}{read\PYZus{}csv}\PY{p}{(}\PY{l+s}{\PYZsq{}}\PY{l+s}{Regulador2.csv}\PY{l+s}{\PYZsq{}}\PY{p}{)}
         \PY{n}{columns} \PY{o}{=} \PY{p}{[}\PY{l+s}{\PYZsq{}}\PY{l+s}{temperatura}\PY{l+s}{\PYZsq{}}\PY{p}{]}
\end{Verbatim}

    \begin{Verbatim}[commandchars=\\\{\}]
{\color{incolor}In [{\color{incolor}69}]:} \PY{c}{\PYZsh{}Mostramos en varias gráficas la información obtenida tras el ensayo}
         \PY{n}{ax2} \PY{o}{=} \PY{n}{datos\PYZus{}it2}\PY{p}{[}\PY{n}{columns}\PY{p}{]}\PY{o}{.}\PY{n}{plot}\PY{p}{(}\PY{n}{figsize}\PY{o}{=}\PY{p}{(}\PY{l+m+mi}{10}\PY{p}{,}\PY{l+m+mi}{5}\PY{p}{)}\PY{p}{,} \PY{n}{ylim}\PY{o}{=}\PY{p}{(}\PY{l+m+mi}{20}\PY{p}{,}\PY{l+m+mi}{100}\PY{p}{)}\PY{p}{,}\PY{n}{title}\PY{o}{=}\PY{l+s}{\PYZsq{}}\PY{l+s}{Modelo matemático del sistema con regulador}\PY{l+s}{\PYZsq{}}\PY{p}{,}\PY{p}{)}
         \PY{n}{ax2}\PY{o}{.}\PY{n}{set\PYZus{}xlabel}\PY{p}{(}\PY{l+s}{\PYZsq{}}\PY{l+s}{Tiempo}\PY{l+s}{\PYZsq{}}\PY{p}{)}
         \PY{n}{ax2}\PY{o}{.}\PY{n}{set\PYZus{}ylabel}\PY{p}{(}\PY{l+s}{\PYZsq{}}\PY{l+s}{Temperatura [ºC]}\PY{l+s}{\PYZsq{}}\PY{p}{)}
\end{Verbatim}

            \begin{Verbatim}[commandchars=\\\{\}]
{\color{outcolor}Out[{\color{outcolor}69}]:} <matplotlib.text.Text at 0xbb0f970>
\end{Verbatim}
        
    \begin{center}
    \adjustimage{max size={0.9\linewidth}{0.9\paperheight}}{modelado_files/modelado_30_1.png}
    \end{center}
    { \hspace*{\fill} \\}
    
    En esta segunda iteracción hemos logrado bajar la sobreoscilación
inicial, pero tenemos mayor error en regimen permanente. Por ello
volvemos a aumentar los valores de \(K_i\) y \(K_d\) siendo los valores
de esta tercera iteracción los siguientes: \(K_p = 6082.6\)
\(K_i=121.64 K_d=60\)

    \subsubsection{Iteracción 3 del
regulador}\label{iteracciuxf3n-3-del-regulador}

    \begin{Verbatim}[commandchars=\\\{\}]
{\color{incolor}In [{\color{incolor}70}]:} \PY{c}{\PYZsh{}Almacenamos en una lista las columnas del fichero con las que vamos a trabajar}
         \PY{n}{datos\PYZus{}it3} \PY{o}{=} \PY{n}{pd}\PY{o}{.}\PY{n}{read\PYZus{}csv}\PY{p}{(}\PY{l+s}{\PYZsq{}}\PY{l+s}{Regulador3.csv}\PY{l+s}{\PYZsq{}}\PY{p}{)}
         \PY{n}{columns} \PY{o}{=} \PY{p}{[}\PY{l+s}{\PYZsq{}}\PY{l+s}{temperatura}\PY{l+s}{\PYZsq{}}\PY{p}{]}
\end{Verbatim}

    \begin{Verbatim}[commandchars=\\\{\}]
{\color{incolor}In [{\color{incolor}73}]:} \PY{c}{\PYZsh{}Mostramos en varias gráficas la información obtenida tras el ensayo}
         \PY{n}{ax3} \PY{o}{=} \PY{n}{datos\PYZus{}it3}\PY{p}{[}\PY{n}{columns}\PY{p}{]}\PY{o}{.}\PY{n}{plot}\PY{p}{(}\PY{n}{figsize}\PY{o}{=}\PY{p}{(}\PY{l+m+mi}{10}\PY{p}{,}\PY{l+m+mi}{5}\PY{p}{)}\PY{p}{,} \PY{n}{ylim}\PY{o}{=}\PY{p}{(}\PY{l+m+mi}{20}\PY{p}{,}\PY{l+m+mi}{180}\PY{p}{)}\PY{p}{,}\PY{n}{title}\PY{o}{=}\PY{l+s}{\PYZsq{}}\PY{l+s}{Modelo matemático del sistema con regulador}\PY{l+s}{\PYZsq{}}\PY{p}{,}\PY{p}{)}
         \PY{n}{ax3}\PY{o}{.}\PY{n}{set\PYZus{}xlabel}\PY{p}{(}\PY{l+s}{\PYZsq{}}\PY{l+s}{Tiempo}\PY{l+s}{\PYZsq{}}\PY{p}{)}
         \PY{n}{ax3}\PY{o}{.}\PY{n}{set\PYZus{}ylabel}\PY{p}{(}\PY{l+s}{\PYZsq{}}\PY{l+s}{Temperatura [ºC]}\PY{l+s}{\PYZsq{}}\PY{p}{)}
\end{Verbatim}

            \begin{Verbatim}[commandchars=\\\{\}]
{\color{outcolor}Out[{\color{outcolor}73}]:} <matplotlib.text.Text at 0xbca3a90>
\end{Verbatim}
        
    \begin{center}
    \adjustimage{max size={0.9\linewidth}{0.9\paperheight}}{modelado_files/modelado_34_1.png}
    \end{center}
    { \hspace*{\fill} \\}
    
    En este caso, se puso un setpoint de 180ºC. Como vemos, la
sobreoscilación inicial ha disminuido en comparación con la anterior
iteracción y el error en regimen permanente es menor. Para intentar
minimar el error, aumentaremos únicamente el valor de \(K_i\). Siendo
los valores de esta cuarta iteracción del regulador los siguientes:
\(K_p = 6082.6\) \(K_i=121.64 K_d=150\)

    \subsubsection{Iteracción 4}\label{iteracciuxf3n-4}

    \begin{Verbatim}[commandchars=\\\{\}]
{\color{incolor}In [{\color{incolor}74}]:} \PY{c}{\PYZsh{}Almacenamos en una lista las columnas del fichero con las que vamos a trabajar}
         \PY{n}{datos\PYZus{}it4} \PY{o}{=} \PY{n}{pd}\PY{o}{.}\PY{n}{read\PYZus{}csv}\PY{p}{(}\PY{l+s}{\PYZsq{}}\PY{l+s}{Regulador4.csv}\PY{l+s}{\PYZsq{}}\PY{p}{)}
         \PY{n}{columns} \PY{o}{=} \PY{p}{[}\PY{l+s}{\PYZsq{}}\PY{l+s}{temperatura}\PY{l+s}{\PYZsq{}}\PY{p}{]}
\end{Verbatim}

    \begin{Verbatim}[commandchars=\\\{\}]
{\color{incolor}In [{\color{incolor}75}]:} \PY{c}{\PYZsh{}Mostramos en varias gráficas la información obtenida tras el ensayo}
         \PY{n}{ax4} \PY{o}{=} \PY{n}{datos\PYZus{}it4}\PY{p}{[}\PY{n}{columns}\PY{p}{]}\PY{o}{.}\PY{n}{plot}\PY{p}{(}\PY{n}{figsize}\PY{o}{=}\PY{p}{(}\PY{l+m+mi}{10}\PY{p}{,}\PY{l+m+mi}{5}\PY{p}{)}\PY{p}{,} \PY{n}{ylim}\PY{o}{=}\PY{p}{(}\PY{l+m+mi}{20}\PY{p}{,}\PY{l+m+mi}{180}\PY{p}{)}\PY{p}{,}\PY{n}{title}\PY{o}{=}\PY{l+s}{\PYZsq{}}\PY{l+s}{Modelo matemático del sistema con regulador}\PY{l+s}{\PYZsq{}}\PY{p}{,}\PY{p}{)}
         \PY{n}{ax4}\PY{o}{.}\PY{n}{set\PYZus{}xlabel}\PY{p}{(}\PY{l+s}{\PYZsq{}}\PY{l+s}{Tiempo}\PY{l+s}{\PYZsq{}}\PY{p}{)}
         \PY{n}{ax4}\PY{o}{.}\PY{n}{set\PYZus{}ylabel}\PY{p}{(}\PY{l+s}{\PYZsq{}}\PY{l+s}{Temperatura [ºC]}\PY{l+s}{\PYZsq{}}\PY{p}{)}
\end{Verbatim}

            \begin{Verbatim}[commandchars=\\\{\}]
{\color{outcolor}Out[{\color{outcolor}75}]:} <matplotlib.text.Text at 0xbc5cfd0>
\end{Verbatim}
        
    \begin{center}
    \adjustimage{max size={0.9\linewidth}{0.9\paperheight}}{modelado_files/modelado_38_1.png}
    \end{center}
    { \hspace*{\fill} \\}
    
    \begin{Verbatim}[commandchars=\\\{\}]
{\color{incolor}In [{\color{incolor} }]:} 
\end{Verbatim}


    % Add a bibliography block to the postdoc
    
    
    
    \end{document}
