
\thispagestyle{empty}
\begin{center}
    \Large
    \vspace{0.9cm}
    \textbf{Resumen}
\end{center}

En los últimos años la tecnología de fabricación aditiva ha experimentado un crecimiento exponencial debido a la liberación de patentes. Comúnmente, se conoce como impresión 3D y cada día tiene más impacto a nivel usuario. La impresora utiliza un filamento de plástico termoplastico para poder crear las piezas, el cual es fundido para poder ser extruido por un fusor. A la vez que el filamento fundido es extruido, el fusor es capaz de moverse en los tres ejes carterianos, el cual va depositando sobre una base las distintas capaz que conforman la pieza final.\\

El filamento usado en las impresoras 3D es creado mediante el proceso de fabricación por extrusión. Este proceso es utilizado habitualmente cuando se fabrican elementos con un perfil constante. A pesar de que la técnica está asentado en la industria, la extrusión de filamento es novedoso, en el que se tienen que tener en cuenta muchos factores que a la hora de fabricar otros perfiles no eran importantes. Aunque son muchos los termoplásticos capaces de ser usados con esta tecnología, los más usados son el ABS y el PLA.\\

Estos factores afectan directamente a la calidad del hilo pudiendo ocasionar serios daños en las impresoras 3D. Hasta ahora, en las líneas de extrusión, los parámetros de fabricación no son importantes una vez la pieza, con perfil constante, ha sido creada, por ello el operador tan solo visualiza los parámetros para que en el instante de fabricación no haya ningún problema. Al fabricar un filamento que va a ser re-fundido es muy importante que todos los datos de produccion queden registrados, puesto que si no cumplen unos requisitos de calidad puede ocasionar daños al usuario final y de esta forma podriamos analizar las posibles causas.\\

Para poder registrar todos los datos que se consideran importantes, este proyecto presenta un sistema de adquisición con la posibilidad de acceder a la información de los parámetros de producción y almacenarla en un medio físico. De esta manera es posible registrar los datos con los que se ha generado el filamento y averiguar cuales son las causas de los fallos. Estos datos son tratados para generar informes detallados de las producciones, y tener un completo historial de la fabricación del filamento, así como un control de calidad del filamento producido.\\

La totalidad del proyecto ha sido realizado en la empresa BQ en el departamento de innovación y robótica, en la división de automatización y materiales. Por tanto, este proyecto proporciona una solución a un problema real que se tiene dentro de una empresa. Así mismo se han abierto distintas líneas de trabajo, en las que se incluye un estudio detallado de como con el paso del tiempo se puede llegar a degradar el PLA y afectar a su correcto funcionamiento.\\

El proyecto presenta una solución a una única línea de extrusión. Pero los planes para el año 2016 es generar un sistema avanzado de supervisión compuesto por varias líneas de extrusión de las que se guardaran todos los datos en una base de datos, y así poder realizar informes, gestionar alarmas y monitorizar todo el sistema de producción.

%\newpage{}
%\begin{center}
%    \Large
%    \vspace{0.9cm}
%    \textbf{Abstract}
%\end{center}
