
\thispagestyle{empty}
\begin{center}
    \Large
    \vspace{0.9cm}
    \textbf{Resumen}
\end{center}

El filamento usado en las impresoras 3D es creado mediante el proceso de fabricación por extrusión. Este proceso es utilizado habitualmente cuando se fabrican elementos con un perfil constante. A pesar de que la técnica está asentada en la industria, la extrusión de filamento es novedoso, ya que se deben de tener en cuenta muchos factores que a la hora de fabricar otros perfiles no eran importantes.\\

Estos factores afectan directamente a la calidad del hilo pudiendo ocasionar serios daños en las impresoras 3D si la calidad es baja. Hasta ahora, en las líneas de extrusión, los parámetros de fabricación son cualitativos, por ello el operador tan solo visualiza los parámetros para que en el instante de fabricación no haya ningún problema. Al fabricar un filamento que va a ser re-fundido es muy importante que todos los datos de produccion, como por ejemplo temperaturas y diámetros del filamento, queden registrados puesto que si no cumplen unos requisitos de calidad puede ocasionar daños a las impresoras 3D y de esta forma podriamos analizar las posibles causas.\\

Este proyecto presenta un sistema de adquisición de datos con la posibilidad de acceder a la información de los parámetros de producción y almacenarla en un medio físico. De esta manera es posible registrar las condiciones con los que se ha generado el filamento y averiguar cuales son las causas de los fallos. Estos datos son tratados para generar informes detallados de las producciones, y tener un completo historial de la fabricación, así como un control de calidad del filamento producido. Con los datos adquiridos por el sistema hemos podido comprobar el funcionamiento de la extrusora, y en caso de detectar un error, hemos sido capaces de encontrar los motivos ayudándonos del mismo. También hemos sido capaces de realizar un estudio de la influencia de un regulador experto en el diámetro del filamento demostrando que los datos obtenidos, con nuestro sistema de adquisición de datos, nos han sido útiles para poder obtener un modelo parcial de nuestro sistema.\\

La totalidad del proyecto ha sido realizado en la empresa BQ. Se han abierto distintas líneas de trabajo, en las que se incluye un estudio detallado de como con el paso del tiempo se puede llegar a degradar el PLA y afectar a su correcto funcionamiento. La información descrita en esta memoria es el primer paso para crear un sistema de gestión de calidad de filamento avanzado, ya que gracias a cómo está planteado el sistema, es facilmente escalable a un mayor número de líneas de extrusión, pudiendo controlar todas las líneas de extrusión desde un lugar centralizado.
\newpage{}
\begin{center}
    \Large
    \vspace{0.9cm}
    \textbf{Abstract}
\end{center}

The filament used in 3D printers is created through the process of fabrication by extrusion. This process is normally used in the fabrication of elements with a constant profile. Despite this technique being well established in the industry, filament extrusion is a novel technique, as it requires taking into account many factors that used to be irrelevant in the fabrication of other profiles.\\

These factors directly affect the quality of the thread, which, when poor, can cause serious damage to 3D printers. To this day, fabrication parameters of extrusion lines are qualitative, hence the operator can only visualise the parameters to make sure there are no issues at the instant of fabrication. When fabricating a filament that is to be remolten, it is very important that all production data such as filament temperatures and diameters remains registered. This is because when filaments do not comply with quality requirements they could cause damage to 3D printers, and in this manner we could analyse the possible causes.\\

This project presents a data acquisition system with the ability to access production parameter information and store it in a physical medium. In this manner, it is possible to register the conditions under which a filament has been generated and find out the causes of errors. This data is treated to generate detailed reports on each production and have a complete fabrication history, as well as a quality control for the produced filament. With the data acquired by the system we have been able to check the running of the extruder, and in the event of an error being detected, we have been able to find the reasons. We have also been able to perform a study on the influence of a regulator specialised on the filament’s diameter, demonstrating that the data obtained with our data acquisition system has been useful in obtaining a partial model for our system.\\

The entirety of this project has been developed at the company BQ. Different lines of work have been opened, among which a detailed study on how the PLA can degrade with the passage of time, affecting its correct functioning. The information described in this report is the first step to create an advanced filament quality management system, as thanks to the way it has been laid-out it is easily scalable to a larger number of extrusion lines, allowing to control all lines of extrusion from a centralised location. \\
